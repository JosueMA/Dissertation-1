\documentclass[12pt, letterpaper]{article}
\usepackage[left=1.00in, right=1.00in, top=1.00in, bottom=1.00in]{geometry}
\usepackage{tikz} \usetikzlibrary{arrows.meta}
\usepackage{caption}
\usepackage{subcaption}
\usepackage{amsmath}

\usepackage[american]{babel}
\usepackage{csquotes}
\usepackage[backend=biber, style=apa]{biblatex}
\DeclareLanguageMapping{american}{american-apa}
\bibliography{../references.bib}

\title{A comparison of the frequentist and Bayesian frameworks in relation to explanatory item response models}
\author{Daniel Furr}
\date{\today}


\begin{document}

% Tikz styles presets ----------------------------------------------------------

% Parameter and data nodes
\tikzstyle{p} = [circle, draw=black, fill=blue!10, thick,
                 inner sep=1pt, minimum size=8mm]
\tikzstyle{d} = [rectangle, draw=black, fill=blue!10, thick,
                 inner sep=1pt, minimum size=8mm]

% Replicate parameter and data nodes
\tikzstyle{pr} = [circle, draw=gray, fill=white, thick,
                  inner sep=1pt, minimum size=8mm]
\tikzstyle{dr} = [rectangle, draw=gray, fill=white, thick,
                  inner sep=1pt, minimum size=8mm]

% Arrow styles for model specification and for generating quantities
\tikzstyle{marrow} = [solid, -{Stealth[length=2mm]}, draw=black]
\tikzstyle{garrow} = [solid, -{Stealth[length=2mm]}, draw=gray]


% Invisisble node (for maintaining equal figure sizes_
\tikzstyle{i} = [circle, draw=none, fill=none, text=white]

% Box for PPCM part of model
\tikzstyle{ppmc}=[fill=red!10, draw=white]

% Boxes and labels to indicate clustering in models
\tikzstyle{items-box}    = [fill = none, draw = black!30!green]
\tikzstyle{items-node}   = [text = black!30!green, anchor = south east]
\tikzstyle{persons-box}  = [fill = none, draw = black!20!orange]
\tikzstyle{persons-node} = [text = black!20!orange, anchor = north west]

\maketitle

\newcommand{\iiiiint}{\int \!\!\! \int \!\!\! \int \!\!\! \int \!\!\! \int}
\newcommand{\iiiiiint}{\int \!\!\! \int \!\!\! \int \!\!\! \int \!\!\! \int \!\!\! \int}


\section{Introduction}

Item response data are cross-classified; that is, any given response to an item is nested both within a person and within an item.
In developing a model for such data, the effects on response probabilities of either or both of persons and items may be regarded as arising from a distribution, and the mean of these distributions may be a function of characteristics of persons or items.
In this way, an item response model may be described as explanatory if it provides estimates of the effects of person and/or item characteristics, in contrast to purely descriptive models that do not include these sorts of effects.

Explanatory Rasch-family item response models are the focus of this chapter. In the frequentist framework, some explanatory Rasch-family models cannot be estimated using the usual marginal maximum likelihood estimation.
In particular, marginal maximum likelihood estimation is infeasible if both the person and item effects are modeled as arising from distributions.
However, when effects of persons but not items are assumed to arise from a distribution, a variety of standard software packages are available to fit such models with relative ease.
There is no such barrier in the Bayesian framework, as models for cross-classified data may be estimated using Markov chain Monte Carlo (MCMC) simulation. The existing software for MCMC tends to be highly flexible but more cumbersome to use.

In this chapter, frequentist and Bayesian approaches to explanatory item response modeling are compared.
Special attention is paid to the predictive inferences that are available under the two frameworks.
Despite the particular context of item response models, the discussion applies to models for clustered data more generally.


\section{A doubly explanatory item response model}

\subsection{General formulation}

A useful model for dichotomous item response data is the Rasch model \parencite{Rasch1960a}:
\begin{equation} \label{eq:base}
	\Pr ( y_{ip} | \theta_p, \delta_i) =
	\frac {\exp(\theta_p - \delta_i)^{y_{ip}}}
	{1 + \exp(\theta_p - \delta_i)}
,\end{equation}
where $y_{ip} = 1$ if person $p$ ($p = 1, \dotsc, P$) responded to item $i$ ($i = 1, \dotsc, I$) correctly and $y_{ip} = 0$ otherwise, $\theta_p$ is the ability of person $p$, and $\delta_i$ is the difficulty of item $i$. The individual instances of $\theta_p$ and $\delta_i$ may be collected into vectors $\theta$ and $\delta$, respectively. This is a ``descriptive'' item response model \parencite{Wilson2004}; it fully accounts for abilities and difficulties, assuming the appropriateness of the model, but does not offer insight into the factors associated with abilities and difficulties.

The model in Equation~\ref{eq:base} may be expanded to a ``person explanatory'' model by decomposing $\theta_p$ as
\begin{equation} \label{eq:theta}
	\theta_p = w_p' \gamma + \zeta_p
,\end{equation}
where $w_p$ is a row from a design matrix $W$ for person-related covariates, $\gamma$ is a vector of regression parameters, and $\zeta_p$ is the residual person ability. The above may be interpreted as a latent regression of ability on covariates $w_p$. $\theta_p$ may be referred to as the composite ability, $w_p' \gamma$ the structured part of ability, and $\zeta_p$ the residual part.

Similarly, decomposing $\delta_i$ as
\begin{equation} \label{eq:delta}
	\delta_i = x_i' \beta + \epsilon_i
\end{equation}
results in an ``item explanatory'' model, in which $x_i$ is a row from a design matrix $X$ for item-related covariates, $\beta$ is a vector of regression parameters, and $\epsilon_i$ is the residual item difficulty. The above is then a latent regression of item difficulty on covariates $x_i$. In parallel with the preceding terminology for ability, $\delta_i$ may be referred to as the composite difficulty, $x_i' \beta$ the structured part of difficulty, and $\epsilon_i$ the residual part.

Equations~\ref{eq:base}, \ref{eq:theta}, and \ref{eq:delta} together form a ``doubly explanatory'' item response model, which incorporates covariates associated with both the persons and items. Note that the model may still serve descriptive purpose as the composite abilities and difficulties remain a part of the model. The final step in formulating the model is to specify distributions for the residuals. In this chapter, normal distributions are assumed,
\begin{equation}
	\zeta_p \sim N(0, \sigma^2)
\end{equation}
and
\begin{equation}
	\epsilon_i \sim N(0, \tau^2)
,\end{equation}
though other choices could be considered. The person and item ``sides'' of the model are specified in directly parallel ways, and much of the discussion that follows will make use of this point.

The model is also presented as a directed graphical model \parencite[for example,][]{dawid1999probabilistic, jordan2004graphical} in Figure~\ref{fig:eirm-model}. In the diagram, unknowns are represented by circles and data are represented by squares. The boxed regions indicate whether the parameters vary over persons, items or neither, and naturally the item responses $y$ vary over both. The direction of the arrows indicates dependence. For example, $\theta$ depends on $\gamma$ and $\zeta$ directly, while it depends indirectly on $\sigma$. This sort of diagram is associated with Bayesian modeling, where all the unknowns may be considered parameters. From a frequentist perspective, the unknowns are a mixture of parameters and latent variables, which will be discussed in greater depth later.

\begin{figure}[btp]
	\centering
	\centering
\begin{tikzpicture}[scale=.75, transform shape]

  \node [p]  (z)  at (0,10) {$\zeta$};
  \node [p]  (s)  at (2,10) {$\sigma$};

  \node [p]  (o)  at (0, 8) {$\theta$};
  \node [p]  (c)  at (2, 8) {$\gamma$};

  \node [d]  (y)  at (0, 6) {$y$};

  \node [p]  (d)  at (0, 4) {$\delta$};

  \node [p]  (e)  at (0, 2) {$\epsilon$};
  \node [p]  (b)  at (2, 4) {$\beta$};
  
  \node [p]  (t)  at (2, 2) {$\tau$};

  \draw [items-box] (-1, 0.75) rectangle (1.25, 7);
  \node [items-node] at (1.25, 0.75) {Items $i$};
  
  \draw [persons-box] (-1.25, 11.25) rectangle (1.00, 5);
  \node [persons-node] at (-1.25, 11.25) {Persons $p$};

  \foreach \from/\to in {s/z, c/o, z/o, o/y, d/y, b/d, t/e, e/d}
    \draw [marrow] (\from) -- (\to);

\end{tikzpicture}
	\caption[The doubly explanatory model]
	{The doubly explanatory model presented as a directed graphical model. Circles represent parameters or latent variables, and squares represent data. Person covariates $w_p$ and item covariates $x_i$ are omitted. The boxed regions indicate whether the parameters vary over persons, items or neither.}
	\label{fig:eirm-model}
\end{figure}


\subsection{Hierarchical Bayes modeling approach}

In Bayesian methodology, the posterior distribution for the parameters is factorized by way of Bayes theorem:
\begin{equation} \label{eq:bayes}
	p(\omega | y, W, X) \propto
	p(\omega)
	p(y | \omega, W, X)
,\end{equation}
which indicates that the posterior distribution is proportional to the product of the prior distribution and the likelihood. In the above, $\omega$ is the set of model parameters. In this chapter, the following terminology for different types of parameters is used.
\begin{enumerate}
	\item \emph{Basic parameters}
	are the foundational parameters. They are plugged into the likelihood directly or affect it indirectly through intermediate parameters. Priors are specified for them.
	\begin{enumerate}
		\item \emph{Exchangeable basic parameters}
		have hierarchical prior distributions. They are exchangeable draws from a distribution, the characteristics of which are determined by hyperparameters. Residuals $\zeta_p$ and $\epsilon_i$ are examples.
		\item \emph{Non-exchangeable basic parameters}
		have non-hierarchical priors. They are not thought of as exchangeable or drawn from a distribution, except in the loose sense that there is some prior distribution. Regression coefficients $\gamma$ and $\beta$ are examples.
	\end{enumerate}
	\item \emph{Intermediate parameters}
	are composites built from basic parameters. They may be included in Bayesian modeling to streamline specifying a model or they may be quantities of genuine interest. They may be plugged into the likelihood in place of basic parameters. They do not have explicit prior distributions, but instead their priors follow from the priors for the basic parameters. Parameters $\theta_p$ and $\delta_i$ are examples
	\item \emph{Hyperparameters}
	are parameters for the distributions for exchangeable basic parameters. They do have prior distributions themselves.
\end{enumerate}
Of the above terms, only ``hyperparameters'' is in general usage, while the remaining types of parameters are not typically distinguished from one another.

The doubly descriptive model has a likelihood (Equation~\ref{eq:base}) based on intermediate parameters $\theta$ and $\delta$, which are in turn built from basic parameters $\gamma$, $\zeta$, $\beta$, and $\epsilon$ (Equations~\ref{eq:theta} and \ref{eq:delta}). $\gamma$ and $\beta$ are non-exchangeable basic parameters, while $\zeta$ and $\epsilon$ are exchangeable basic parameters whose priors depend on hyperparameters $\sigma$ and $\tau$, respectively.
The prior distribution in Equation~\ref{eq:bayes} may be rewritten as
\begin{equation} \label{eq:prior}
	p(\omega) =
	p(\gamma) p(\sigma)
	\left [
		\prod_{p=1}^P p(\zeta_p | \sigma)
	\right ]
	p(\beta) p(\tau)
	\left [
		\prod_{i=1}^I p(\epsilon_i | \tau)
	\right ]
\end{equation}
if independent priors are specified, which is the usual case. No prior is included for $\theta$ and $\delta$ as they are wholly determined from the basic parameters.
%The joint prior $p(\gamma, \beta, \tau, \sigma, \zeta, \epsilon)$ may be written as the product of individual priors if the priors are specified as independent, which is the usual case.
The likelihood part of Equation~\ref{eq:bayes} may be rewritten in terms of basic parameters as
\begin{equation} \label{eq:bayes-likelihood-alt}
	p(y | \omega, W, X) =
%	p(y | W, X, \gamma, \beta, \zeta, \epsilon) =
	\prod_{p=1}^P \prod_{i=1}^I
	\Pr(y_{ip} | w_p, x_i, \zeta_p, \gamma, \epsilon_i, \beta)
,\end{equation}
%where $\omega = \{\gamma, \beta, \sigma\}
or in terms of intermediate parameters as
\begin{equation} \label{eq:bayes-likelihood}
	p(y | \omega, W, X) =
	p(y | \theta, \delta) =
	\prod_{p=1}^P \prod_{i=1}^I \Pr(y_{ip} | \theta_p, \delta_i)
.\end{equation}
Given that both the prior and the likelihood may be specified ignoring the intermediate parameters $\theta$ and $\delta$, it is clear that they are redundant. For many applications, however, they have useful interpretations, and for that reason estimation of their posteriors may be desired. Further, the posterior distributions of $\theta$ and $\delta$ can be estimated easily from the posterior draws of the basic parameters.

The posterior for a single parameter, marginal in regards to all other parameters, may be obtained by integrating the full joint posterior over all other parameters. Let $D = \{ y, W, X \}$, represent the full data. Then,
\begin{equation}
	p(\sigma | D) =
	\iiiiint
		p(\zeta, \gamma, \sigma, \epsilon, \beta, \tau | D)
	~d \zeta d \gamma d \epsilon d \beta d \tau
\end{equation}
is the posterior for the standard deviation of the ability residuals. The mean and standard deviation of the marginal posterior for a parameter may be taken to represent a point estimate and standard error. Further, the joint posterior of a subset of parameters, $p(\beta, \zeta_p, \gamma, \epsilon_i | D)$ for example, likewise may be obtained by integrating out the other parameters. Despite the high-dimensional integral involved, these quantities are readily available from Monte Carlo simulation by simply ignoring the draws for the parameters to be integrated out, and so no special effort is required to obtain them.

The model could equivalently be specified using hierarchical centering \parencite{gelfand1995efficient} by replacing the preceding prior with
\begin{equation}
	p(\omega) =
	p(\gamma) p(\sigma)
	\left [
		\prod_{p=1}^P p(\theta_p | w_p, \gamma, \sigma)
	\right ]
	p(\beta) 	p(\tau)
	\left [
		\prod_{i=1}^I p(\delta_i | x_i, \beta, \tau)
	\right ]
\end{equation}
where
$p(\theta_p | w_p, \gamma, \sigma) = \mathrm{N}(w_p \gamma, \sigma^2)$ and
$p(\delta_i | x_i, \beta, \tau)    = \mathrm{N}(x_i \beta, \tau^2)$,
respectively. The likelihood is still specified as in Equation~\ref{eq:bayes-likelihood}. In this formulation, residuals $\zeta$ and $\epsilon$ are omitted altogether, and $\theta$ and $\delta$ are treated as exchangeable (conditional on covariates) basic parameters rather than as intermediate parameters. Depending on the data and on the algorithm used, this formulation may improve the efficiency of the MCMC simulation. Because this chapter includes discussion of inferences related to the residuals, the ``decentered'' formulation described before is preferred.


\subsection{Frequentist modeling approach}

In the frequentist approach, only the non-exchangeable basic parameters ($\gamma$ and  $\beta$) and the hyperparameters ($\sigma$ and $\tau$) are treated as parameters to be estimated. In this framework, marginal maximum likelihood may be used to estimate the model, which involves marginalizing the residuals out of the likelihood. The marginal likelihood is
\begin{equation} \label{eq:marginal-likelihood}
p(y | \gamma, \beta, \sigma, \tau) =
\int \cdots \iint \cdots \int
	\prod_{p=1}^P \prod_{i=1}^I
	\left [
		\Pr(y_{ip} | \zeta_p, \gamma, \epsilon_i, \beta)
		p(\zeta_p | \sigma)
		p(\epsilon_i | \tau)
	\right ]
~d \zeta_1 \ldots d \zeta_P d \epsilon_1 \ldots d \epsilon_I
,\end{equation}
in which the probability of a response is marginal over the distributions for person and item residuals. Point estimates $\hat\gamma$, $\hat\sigma$, $\hat\beta$, and $\hat\tau$ are obtained by maximizing this likelihood

Within this framework, the exchangeable parameters $\zeta$ and $\epsilon$ are called latent variables or random effects because parameters cannot have distributions. Rather than obtain direct estimates for random effects, marginal maximum likelihood estimation obtains estimates for the parameters of their distributions only, in this case, $\hat \sigma$ and $\hat\tau$. The non-exchangeable basic parameters $\gamma$ and $\beta$ are sometimes referred to as ``fixed-effects.''
%\emph{``Can choose whether to treat $\zeta_p$ or $\epsilon_i$ as latent variables (random effects) or just consider their marginal likelihood, for example, to just get a covariance structure.'' Ref: Molenberghs and Verbeke. Hard to find because they wrote a million articles together.}

A model of this kind may be formulated in the generalized linear mixed model framework. The response variable, conditional on covariates and so-called random effects, is specified as arising from a Bernoulli distribution:
\begin{equation}
	y_{ip} | w_p, x_i, \zeta_p, \epsilon_i \sim \mathrm{Bernoulli}(\pi_{ip})
.\end{equation}
Then the model may be written in terms of an inverse link function
\begin{equation}
	\pi_{ip} =
	\Pr(y_{ip} = 1 | w_p, x_i, \zeta_p, \epsilon_i) =
	\mathrm{logit}^{-1}[\eta_{ip}]
\end{equation}
and a linear predictor
\begin{equation}
	\eta_{ip} =
	(w_p'\gamma + \zeta_p) -
	(x_i'\beta + \epsilon_i)
.\end{equation}
Because the random-effects $\zeta_p$ and $\epsilon_i$ are not nested, the model may be described as a crossed-random effects model. Such a model is difficult to estimate efficiently via marginal maximum likelihood because the integrals in Equation~\ref{eq:marginal-likelihood} do not factorize as they do with nested random effects. The result is an $I \times P$ dimensional integral, though \textcite{rasbash1994efficient} describe a means of reducing this to an $I + 1$ dimensional integral.

%double integration over the latent distributions (see Equation~\ref{eq:mml-likelihood}). The Laplace approximation may be employed to make the math tractable, but this approach has known shortcomings \parencite{Joe2008}.

%Though not directly estimated, post-hoc estimates for $\hat\zeta_p$ and $\hat\epsilon_i$ may be obtained by finding the mean or mode of $p(\yp|\zeta_p) p(\zeta_p|\hat\sigma)$ or $p(\yi|\epsilon_i) p(\epsilon_i|\hat\tau)$, which are referred to as ``empirical Bayes'' estimates. Standard errors are available for these, though they are calculated as though the parameter estimates are known, rather than random, quantities.
%
%Also available only in post-analysis are estimates
%$\hat\theta_p = w_p' \hat\gamma + \hat\zeta_p$ and
%$\hat\delta_i = x_i' \hat\beta + \hat\epsilon_i$,
%where $\hat\zeta_p$ and $\hat\epsilon_i$ are empirical Bayes estimates. Standard errors are available for these quantities, though they also are calculated as if the parameter estimates are known quantities.


\subsection{Special cases}
\label{sec:special-cases}

Many dichotomous item response models are special cases of the doubly explanatory model that arise from restrictions placed on the composite abilities and difficulties. For example, the Rasch model \parencite{Rasch1960a} as fit by marginal maximum likelihood estimation \parencite{bock1981marginal} can be written as
\begin{equation}
	\Pr ( y_{ip} | \theta_p, \delta_i) =
	\frac {\exp(\theta_p - \delta_i)^{y_{ip}}}
	{1 + \exp(\theta_p - \delta_i)}
\end{equation}
\begin{equation}
	\theta_p = \zeta_p
\end{equation}
\begin{equation}
	\delta_i = x_i' \beta
,\end{equation}
where $X$ is an $I \times I$ identity matrix ($I_I$) and $\beta$ is a vector of length $I$, such that $\delta_i = \beta_i$. In other words, $\delta_i$ is set equal to the (unstructured) structural part of item difficulty, while $\theta_p$ is set equal to the ability residuals.

In the Bayesian approach, the posterior for this Rasch model variant is given by
\begin{equation}
	p(\theta, \sigma, \delta | y) \propto
	\left [
		p(\delta) 	p(\sigma)
		\prod_{p=1}^P p(\theta_p | \sigma)
	\right ]
	\left [
		\prod_{p=1}^P \prod_{i=1}^I
		\Pr ( y_{ip} | \theta_p, \delta_i)
	\right ]
,\end{equation}
in which the left hand bracketed quantity is the prior and and the right hand quantity is the likelihood. The marginal likelihood for the frequentist approach is
\begin{equation}
	p(y | \sigma, \delta) =
	\prod_{p=1}^P
	\int
		\prod_{i=1}^I
		\Pr(y_{ip} | \theta_p, \delta_i)
		p(\theta_p | \sigma)
	~d \theta_p
.\end{equation}
The single-dimensional integration is simpler than the $I \times P$ dimensional integral in Equation~\ref{eq:marginal-likelihood} and may be approximated using adaptive quadrature \parencite{rabe2002reliable}.

\begin{table}
	\centering
	\begin{tabular}{lccc}
		\hline
		Model	& $\theta_p$ & $\delta_i$ & Notes \\ \hline
		MML Rasch
			& $\zeta_p$ & $x_i'\beta$ & $X = I_I$ \\
		JML Rasch
			& $w_p' \gamma$ & $x_i'\beta$ & $W = I_{P-1}$, $X = I_I$ \\
		Random item Rasch
			& $\zeta_p$ & $\epsilon_i$ &  \\
		Latent regression
			& $w_p' \gamma + \zeta_p$ & $x_i'\beta$ & $X = I_I$ \\
		Linear logistic test
			& $\zeta_p$ & $x_i'\beta$ &  \\
		Linear logistic test with error
			& $\zeta_p$ & $x_i'\beta + \epsilon_i$ &  \\
		Doubly explanatory
			& $w_p' \gamma + \zeta_p$ &  $x_i'\beta + \epsilon_i$ & \\
		\hline
	\end{tabular}
	\caption[Specification of several special cases of the doubly explanatory model]
	{Specification of several special cases of the doubly explanatory model.}
	\label{tab:special-cases}
\end{table}

Other special cases arise from different choices of restrictions placed on the composite abilities and difficulties, and these are summarized in Table~\ref{tab:special-cases}. As mentioned earlier, the Rasch model as fit by joint maximum likelihood estimation \parencite[for example,][]{embretson2000item} includes only the structured parts of ability and difficulty with identity matrices for $W$ and $X$ (one difficulty or ability parameter must be constrained for identifiability). In contrast, the random item Rasch model \parencite[for example,][]{DeBoeck2008} has only the residual parts for both sides (a model intercept must be added). The latent regression item response model \parencite{mislevy1985estimation, adams1997multilevel} includes both parts of the composite ability and the structured part of item difficulty, where $X$ is an identity matrix. The linear logistic test model (LLTM) \parencite{fischer1973linear}, has the residual part for ability and the structured part for difficulty. Its extension, the linear logistic test model with error (LLTM-E) \parencite[for example,][]{mislevy1988exploiting, Janssen2004}, adds an item difficulty residual.


\section{Estimated and predicted quantities}

Several quantities from the model may be of interest, whether they are estimated directly or obtained after estimation. At the macro-level, $\gamma$ represents the effects of the person covariates, and $W \gamma$ together with $\sigma$ describes the conditional distribution for person abilities. Likewise, $\beta$ represents the effects of the item covariates, and $X \beta$ together with $\tau$ describes the conditional distribution for item difficulties. Depending on the choice of either a frequentist and Bayesian framework, the maximum likelihood estimates $\hat \gamma$ and $\hat \beta$ or posterior distributions $p(\gamma | D)$ and $p(\beta | D)$ will be obtained for these parameters.

For some applications, such as measurement ``per se'', the specific persons and items will be of interest. This is the case when, for example, measurements are needed for person abilities and a Wright map \parencite{wilson2004constructing} is used in interpreting them in relation to the item difficulties. In this case, attention will be placed on $\theta$ and $\delta$, though $\zeta$ and $\epsilon$ may be of interest in the identification of outliers. These are within-sample quantities; that is, the estimation sample contains a person $p$ who is associated with $\zeta_p$ and $\theta_p$ and also an item $i$ that is associated with $\epsilon_i$ and $\delta_i$.

There may be a (real or hypothetical) person $p'$ not represented in the estimation data. This out-of-sample person has a covariate vector $w_{p'}$ and is associated with parameters $\tilde \zeta_{p'}$ and $\tilde \theta_{p'}$, none of which play a role in fitting the model. Likewise, an out-of-sample item $i'$ associated with $x_{i'}$, $\tilde \epsilon_{i'}$, and $\tilde \delta_{i'}$ may be envisioned. Inferences for these out-of-sample quantities may be obtained from the fitted model.

Inferences for the within-sample quantities $\theta_p$, $\delta_i$, $\zeta_p$, and $\epsilon_i$ are called predictions in the frequentist framework because they are random variables (and not parameters) that are not directly estimated from the model. The same inferences are estimates in a Bayesian setting where $\zeta_p$ and $\epsilon_i$ are drawn from the posterior and $\theta_p$ $\delta_i$ are functions of parameters drawn from the posterior. Inferences for the out-of-sample quantities $\tilde \theta_{p'}$, $\tilde \delta_{i'}$, $\tilde \zeta_{p'}$, and $\tilde \epsilon_{i'}$ are considered predictions in either case.

Lastly, inferences may be made regarding new responses, which are always considered predictions. A new response may be conceived as arising from a within-sample person to a within-sample item, indicated by $\tilde y_{ip}$. This is, in other words, simply a model-predicted response for an existing observation. Several possibilities exist for out-of-sample responses: $\tilde y_{i'p}$ represents a new response from a within-sample person to an out-of-sample item, $\tilde y_{ip'}$ represents a new response from an out-of-sample person to a within-sample item, and $\tilde y_{i'p'}$ represents a new response when both the associated item and person are out-of-sample. 


\subsection{Inferences for residuals}

Starting with the Bayesian perspective, the posterior for residual $\zeta_p$ is
%\begin{equation}
%	p(\zeta_p | D)
%	= \int p(\zeta_{p} | D, \sigma) p(\sigma | D) ~d\sigma
%,\end{equation}
\begin{equation}
	p(\zeta_p | D) =
	\iiiiiint
		p(\zeta, \gamma, \sigma, \epsilon, \beta, \tau | D)
	~d \zeta_{-p} d \gamma d \sigma d \epsilon d \beta d \tau
,\end{equation}
where $\zeta_{-p}$ is the vector $\zeta$ omitting $\zeta_p$. This is simply the full posterior integrating out all other parameters and its distribution be approximated in MCMC simulation simply by $\zeta_p^s$, where $s = 1 \ldots S$ indexes the draws from the simulation. The distribution for the residual of a new person, $\tilde \zeta_{p'}$, is
\begin{equation}
	p(\tilde \zeta_{p'} | D)
	= \int p(\tilde \zeta_{p'} | \sigma) p(\sigma | D) ~d\sigma
,\end{equation}
which is referred to as a mixed predictive distribution \parencite{Gelman1996}. It may be approximated by taking random draws for $\tilde \zeta_{p'}^s$ from its prior, $p(\tilde \zeta_{p'} | \sigma^s)$. On the item side, the parallel quantities are
%\begin{equation}
%	p(\epsilon_i | D)
%	= \int p(\tilde \epsilon_i | D, \tau) p(\tau | D) ~d\tau
%\end{equation}
\begin{equation}
	p(\epsilon_i | D) =
	\iiiiiint
		p(\zeta, \gamma, \sigma, \epsilon, \beta, \tau | D)
	~d \zeta d \gamma d \sigma d \epsilon_{-i} d \beta d \tau
,\end{equation}
and
\begin{equation}
	p(\tilde \epsilon_{i'} | D)
	= \int p(\tilde \epsilon_{i'} | \tau) p(\tau | D) ~d\tau
.\end{equation}
%The posteriors for $\zeta_p$ and $\epsilon_i$ are directly estimated within the MCMC simulation, given that they are parameters, whereas the mixed predictive distributions for $\tilde \zeta_{p'}$ and $\tilde \epsilon_{i'}$ are easily obtained by random draws from their priors within the simulation. 
\textcite{marshall2007identifying} have recommended using mixed predictive distributions like $p(\tilde \zeta_{p'} | D)$ and $p(\tilde \epsilon_{i'} | D)$ to detect outlying residuals.

From the frequentist perspective, ``empirical Bayes'' predictions for residuals may be obtained post-estimation. The empirical Bayes mean prediction for $\zeta_p$ is
\begin{equation}
	\hat \zeta_p^\mathrm{EB} =
	\int 
		\zeta_p ~p(\zeta_p |D, \hat \gamma, \hat \sigma 
		                    \hat \beta, \hat \tau) 
	~d\zeta_p
,\end{equation}
where $p(\zeta_p |D, \hat \gamma, \hat \sigma, \hat \beta, \hat \tau)$ is the conditional posterior
\begin{equation}
	p(\zeta_p |D, \hat \gamma, \hat \sigma,  \hat \beta, \hat \tau)  \propto
	p(\zeta_p| \hat \sigma)
	p(y_p | w_p, X, \hat \gamma, \zeta_p, \hat \beta, \hat \tau)
.\end{equation}
The rightmost quantity in the above is the likelihood conditional on $\zeta_p$ but marginal in regard to $\epsilon$:
%\begin{equation}
%	p(y_p | w_p, X, \hat \gamma, \zeta_p, \hat \beta, \hat \tau) =
%	\int \cdots \int
%		p(\epsilon_1 \ldots \epsilon_I | \hat \tau)
%		p(y_p | w_p, X, \hat \gamma, \hat \zeta_p \hat \beta, 
%		        \epsilon_1 \ldots \epsilon_I)
%	~d \epsilon_1 \ldots \epsilon_I
%.\end{equation}
\begin{equation}
	p(y_p | w_p, X, \hat \gamma, \zeta_p, \hat \beta, \hat \tau) =
	\int
		p(\epsilon | \hat \tau)
		p(y_p | w_p, X, \hat \gamma, \zeta_p, \hat \beta, \epsilon)
	~d \epsilon
.\end{equation}
The above form for the empirical Bayes prediction is more complicated than usual owing to the need to integrate out the $\epsilon$ vector, which arises from the model being for cross-classified data. Instead of the empirical Bayes mean prediction, the modal prediction may be obtained by finding the mode of the conditional posterior. Of course, the empirical Bayes prediction for either the mean or mode of $\epsilon_i$ may be written in a way parallel to that for $\zeta_p$. The main difference between the frequentist empirical Bayes approach and actual Bayesian approach is the propagation of uncertainty; while the frequentist approach treats the model parameters as known when obtaining the prediction, the Bayesian approach incorporates the residuals as a part of the full posterior. Lastly, frequentists may take $p(\tilde \zeta_p | \hat \sigma)$ and $p(\tilde \epsilon_i | \hat \tau)$ as representing the distributions for new instances of the residuals, and as both have a mean of zero, zero may be assigned as the point predictions for residuals for new persons or new items.


\subsection{Inferences for composites}

Returning to the Bayesian perspective, the posterior for composites like $p(\theta_p | D)$ are easily approximated from the posterior draws of MCMC simulation:
\begin{equation}
	\theta_p^s
	= w_p' \gamma^s + \zeta_p^s
.\end{equation}
The posterior for a new composite ability, $p(\tilde \theta_{p'} | D)$, is approximated by the empirical distribution of
\begin{equation}
	\tilde \theta_{p'}^s
	= w_{p'}' \gamma^s + \tilde \zeta_{p'}^s
\end{equation}
where $w_{p'}$ is the covariate vector for the new person and $\tilde \zeta_{p'}^s$ is as given above. In a parallel way, $p(\delta_i | D)$ and $p(\tilde \delta_{i'} | D)$ may be approximated by the distributions of
\begin{equation}
	\delta_i^s
	= x_i' \beta^s + \epsilon_i^s
.\end{equation}
and
\begin{equation}
	\tilde \delta_{i'}^s
	= x_{i'}' \beta^s + \tilde \epsilon_{i'}^s
,\end{equation}
respectively.

In the frequentist perspective, a prediction for an in-sample composite ability is a combination of the regression prediction and the empirical Bayes estimate for the residual:
\begin{equation}
	\hat \theta_p
	= w_p' \hat \gamma + \hat \zeta_p^\mathrm{EB}
.\end{equation}
For an out-of-sample composite ability, the residual part of the prediction may be set to zero (the mean of residuals):
\begin{equation}
	\tilde \theta_{p'}
	= w_{p'}' \hat \gamma^s
.\end{equation}
The equivalent quantities on the item side are
\begin{equation}
	\hat \delta_i
	= x_i' \hat \beta + \hat \epsilon_i^\mathrm{EB}
\end{equation}
and
\begin{equation}
	\tilde \delta_{i'}
	= x_{i'}' \hat \beta^s
.\end{equation}
As with the predictions for residuals, each of these are point estimates and do not involve the propagation of uncertainty realized in Bayesian modeling.


\subsection{Inferences for responses}

Returning again to the Bayesian framework, the posterior predictive distribution \parencite{rubin1984bayesianly} for new a response $\tilde y_{ip}$ from a within-sample person and item is
\begin{align}
	p(\tilde y_{ip} | D)
	&= \iint
		\Pr (\tilde y_{ip} | \theta_p, \delta_i)
		p(\theta_p, \delta_i | D)
	~d\theta_p d\delta_i \\
	&= \iiiint
		\Pr (\tilde y_{ip} | w_p, x_i, \gamma, \zeta_p, \beta, \epsilon_i)
		p(\gamma, \zeta_p, \beta, \epsilon_i | D)
	~d\gamma d\zeta_p d\beta d\epsilon_i
.\end{align}
The predictive distribution for a new response arising from an out-of-sample person and out-of-sample item is
\begin{align}
	p(\tilde y_{i'p'} | D)
	&= \iint
		\Pr (\tilde y_{i'p'} | \tilde \theta_{p'}, \tilde \delta_{i'})
		p(\tilde \theta_{p'}, \tilde \delta_{i'} | D)
	~d\tilde \theta_{p'} d \tilde \delta_{i'} \\
	&= \iiiint
		\Pr (\tilde y_{i'p'} | w_{p'}, x_{i'}, \gamma, \tilde \zeta_{p'}, 
		                       \beta, \tilde \epsilon_{i'})
		p(\gamma, \tilde \zeta_{p'}, \beta, \tilde \epsilon_{i'} | D)
	~d\gamma d \tilde \zeta_{p'} d \beta d \tilde \epsilon_{i'}
,\end{align}
where
$p(\tilde \theta_{p'}, \tilde \delta_{i'} | D)$ 
is the joint mixed predictive distribution for $\tilde \theta_{p'}$ and $\tilde \delta_{i'}$, and 
$p(\gamma, \tilde \zeta_{p'}, \beta, \tilde \epsilon_{i'} | D)$
includes the mixed predictive distributions for $\tilde \zeta_{p'}$ and $\tilde \epsilon_{i'}$, as described previously.
In MCMC simulation, $p(\tilde y_{ip}^s | D)$ may be obtained as a random draw from
$\mathrm{Bernoulli}(\mathrm{logit}^{-1}(\theta_{p}^s - \delta_{i}^s))$,
and likewise $p(\tilde y_{i'p'}^s | D)$ may be obtained as random draw from
$\mathrm{Bernoulli}(\mathrm{logit}^{-1}(\tilde \theta_{p'}^s - \tilde \delta_{i'}^s))$.

Figure~\ref{fig:ppmc-models} shows four ways of making inferences for new responses. On the left side of each panel is a graphical representation of the model, similar to the one shown earlier, though the boxed regions indicating which parameters vary over persons and which vary over items are omitted for simplicity.
On the right side of each is a shaded region for the out-of-sample predictions.
Figure~\ref{subfig:ppmc-same-both} shows that the posterior distribution for $\tilde y_{ip}$, a new response from an in-sample person and in-sample item, arises directly from an existing $\theta_p$ and $\delta_i$ pair. In this way, it is clear that $\tilde y_{ip}$ is closely related to the observed $y_{ip}$.
Figure~\ref{subfig:ppmc-new-both} shows that the posterior for $\tilde y_{i'p'}$ arises from the mixed predictive distributions for $\tilde \theta_{p'}$ and $\tilde \delta_{i'}$. Further, it depicts how the various predictive distributions are influenced by the posteriors for the modeled parameters.
Lastly, predictive distributions for responses from a new person to an in-sample item, $p(\tilde y_{ip'} | D)$, as well as responses from an in-sample person to a new item, $p(\tilde y_{i'p} | D)$, may be obtained by mixing and matching posterior and mixed predictive distributions as needed, as shown in Figures~\ref{subfig:ppmc-new-persons} and \ref{subfig:ppmc-new-items}.

\begin{figure}[btp]
	\centering
	\begin{subfigure}[b]{.4\textwidth}
		\centering
\begin{tikzpicture}[scale=.75, transform shape]

  \filldraw [ppmc] (3.25, 1.25) rectangle (4.75, 10.75);
  \node [i]  at (0, 12) {~}; % Force some vertical space with invisible node.

  \node [p]  (z)  at (0,10) {$\zeta$};
  \node [p]  (s)  at (2,10) {$\sigma$};

  \node [p]  (o)  at (0, 8) {$\theta$};
  \node [p]  (c)  at (2, 8) {$\gamma$};

  \node [d]  (y)  at (0, 6) {$y$};
  \node [dr] (yr) at (4, 6) {$\tilde y$};

  \node [p]  (d)  at (0, 4) {$\delta$};

  \node [p]  (e)  at (0, 2) {$\epsilon$};
  \node [p]  (b)  at (2, 4) {$\beta$};

  \node [p]  (t)  at (2, 2) {$\tau$};

  \foreach \from/\to in {s/z, c/o, z/o, o/y, d/y, b/d, t/e, e/d}
    \draw [marrow] (\from) -- (\to);

  \foreach \from/\to in {d/yr, o/yr}
    \draw [garrow] (\from) -- (\to);

\end{tikzpicture}

		\caption{Within-sample persons and items ($\tilde y_{ip}$)}
		\label{subfig:ppmc-same-both}
	\end{subfigure}
	~
	\begin{subfigure}[b]{.4\textwidth}
		\centering
\begin{tikzpicture}[scale=.75, transform shape]

  \filldraw [ppmc] (3.25, 1.25) rectangle (4.75, 10.75);
  \node [i]  at (0, 12) {~}; % Force some vertical space with invisible node.

  \node [p]  (z)  at (0,10) {$\zeta$};
  \node [p]  (s)  at (2,10) {$\sigma$};
  \node [pr] (zr) at (4,10) {$\tilde \zeta$};

  \node [p]  (o)  at (0, 8) {$\theta$};
  \node [p]  (c)  at (2, 8) {$\gamma$};
  \node [pr] (or) at (4, 8) {$\tilde \theta$};

  \node [d]  (y)  at (0, 6) {$y$};
  \node [dr] (yr) at (4, 6) {$\tilde y$};

  \node [p]  (d)  at (0, 4) {$\delta$};
  \node [p]  (b)  at (2, 4) {$\beta$};
  \node [pr] (dr) at (4, 4) {$\tilde \delta$};

  \node [p]  (e)  at (0, 2) {$\epsilon$};
  \node [p]  (t)  at (2, 2) {$\tau$};
  \node [pr] (er) at (4, 2) {$\tilde \epsilon$};

  \foreach \from/\to in {s/z, c/o, z/o, o/y, d/y, b/d, t/e, e/d}
    \draw [marrow] (\from) -- (\to);

  \foreach \from/\to in {s/zr, zr/or, c/or, or/yr, t/er, er/dr, b/dr, dr/yr}
    \draw [garrow] (\from) -- (\to);

\end{tikzpicture}

		\caption{Out-of-sample persons and items ($\tilde y_{i'p'}$)}
		\label{subfig:ppmc-new-both}
	\end{subfigure}
	~
	\begin{subfigure}[b]{.4\textwidth}
		\centering
\begin{tikzpicture}[scale=.75, transform shape]

  \filldraw [ppmc] (3.25, 1.25) rectangle (4.75, 10.75);
  \node [i]  at (0, 12) {~}; % Force some vertical space with invisible node.

  \node [p]  (z)  at (0,10) {$\zeta$};
  \node [p]  (s)  at (2,10) {$\sigma$};
  \node [pr] (zr) at (4,10) {$\tilde \zeta$};

  \node [p]  (o)  at (0, 8) {$\theta$};
  \node [p]  (c)  at (2, 8) {$\gamma$};
  \node [pr] (or) at (4, 8) {$\tilde \theta$};

  \node [d]  (y)  at (0, 6) {$y$};
  \node [dr] (yr) at (4, 6) {$\tilde y$};

  \node [p]  (d)  at (0, 4) {$\delta$};
  \node [p]  (b)  at (2, 4) {$\beta$};

  \node [p]  (e)  at (0, 2) {$\epsilon$};
  \node [p]  (t)  at (2, 2) {$\tau$};

  \foreach \from/\to in {s/z, c/o, z/o, o/y, d/y, b/d, t/e, e/d}
    \draw [marrow] (\from) -- (\to);

  \foreach \from/\to in {s/zr, zr/or, c/or, or/yr, d/yr}
    \draw [garrow] (\from) -- (\to);

\end{tikzpicture}

		\caption{Out-of-sample persons and within-sample items ($\tilde y_{ip'}$)}
		\label{subfig:ppmc-new-persons}
	\end{subfigure}
	~
	\begin{subfigure}[b]{.4\textwidth}
		\centering
\begin{tikzpicture}[scale=.75, transform shape]

  \filldraw [ppmc] (3.25, 1.25) rectangle (4.75, 10.75);
  \node [i]  at (0, 12) {~}; % Force some vertical space with invisible node.

  \node [p]  (z)  at (0,10) {$\zeta$};
  \node [p]  (s)  at (2,10) {$\sigma$};

  \node [p]  (o)  at (0, 8) {$\theta$};
  \node [p]  (c)  at (2, 8) {$\gamma$};

  \node [d]  (y)  at (0, 6) {$y$};
  \node [dr] (yr) at (4, 6) {$\tilde y$};

  \node [p]  (d)  at (0, 4) {$\delta$};
  \node [p]  (b)  at (2, 4) {$\beta$};
  \node [pr] (dr) at (4, 4) {$\tilde \delta$};

  \node [p]  (e)  at (0, 2) {$\epsilon$};
  \node [p]  (t)  at (2, 2) {$\tau$};
  \node [pr] (er) at (4, 2) {$\tilde \epsilon$};

  \foreach \from/\to in {s/z, c/o, z/o, o/y, d/y, b/d, t/e, e/d}
    \draw [marrow] (\from) -- (\to);

  \foreach \from/\to in {o/yr, t/er, er/dr, b/dr, dr/yr}
    \draw [garrow] (\from) -- (\to);

\end{tikzpicture}

		\caption{Within-sample persons and out-of-sample items ($\tilde y_{i'p}$)}
		\label{subfig:ppmc-new-items}
	\end{subfigure}
	\caption[Predictive distributions of various forms for responses under the doubly explanatory model]
	{Predictive distributions of various forms for responses under the doubly explanatory model. Circles represent parameters and squares represent data. The shaded region indicates predictive quantities that are not involved in the estimation. Covariates $W$ and $X$ are omitted.}
	\label{fig:ppmc-models}
\end{figure}

In the frequentist perspective, the predicted probabilities for a correct response are based on point estimates of model parameters, but are otherwise similar to the Bayesian predictions. For a new response from a within-sample person-item pair, the predicted probability of a correct response is
\begin{equation}
	\Pr(\tilde y_{ip} = 1| w_p, x_i, \hat \gamma, \hat \sigma, \hat \beta, \hat \tau) =
	\iint 
		\Pr(\tilde y_{ip} = 1| w_p, x_i, \hat \gamma, \zeta_p, \hat \beta, \epsilon_i)
		p(\zeta_p, \epsilon_i | D, \hat \gamma, \hat \sigma, \hat \beta, \hat \tau)
	~d \zeta_p d \epsilon_i
.\end{equation}
Like empirical Bayes predictions, it uses the conditional posterior for $\zeta_p$ and $\epsilon_i$. This corresponds to the cluster-averaged expectation for generalized linear mixed models described by \textcite{skrondal2009prediction}, except that the prediction given here marginalizes over the posterior for two sets of residuals rather than just one. The predicted probability for a new response from a new person to a new item is
\begin{equation}
	\Pr(\tilde y_{i'p'} = 1| w_{p'}, x_{i'}, \hat \gamma, \hat \beta, \hat \tau, \hat \sigma) =
	\iint 
		\Pr(\tilde y_{i'p'} = 1| w_{p'}, x_{i'}, \hat \gamma, \hat \beta, \zeta_{p'}, \epsilon_{i'})
		p(\zeta_{p'}, \epsilon_{i'} | \hat \tau, \hat \sigma)
	~d \zeta_{p'} d \epsilon_{i'}
,\end{equation}
which uses the prior for $\zeta_p$ and $\epsilon_i$. This corresponds to what \textcite{skrondal2009prediction} refer to as the population-averaged expectation, again with the exception that two sets of residuals are involved here. Lastly, predictions for new responses of the form $\tilde y_{i'p}$ and $\tilde y_{ip'}$ are obtained by mixing the use of posterior and prior distributions for the residuals.

%
%
%\begin{equation}
%	%\Pr(\tilde y_{ip} = 1 | \hat \theta_p, \hat \delta_i)
%	%= \mathrm{logit}^{-1}(\hat \theta_p - \hat \delta_i)
%	\Pr(\tilde y_{ip}=1 | D, \hat \gamma, \hat \beta, \hat \tau, \hat \sigma)
%	\iint \Pr(\tilde y_{ip}=1 | \theta_p, \delta_i)
%		p(\theta_p, \delta_i | w_p, x_i, \hat \gamma, \hat \beta, \hat \tau, \hat \sigma)
%	~d \zeta_p d \epsilon_i
%\end{equation}
%or equivalently
%\begin{equation}
%	\Pr(\tilde y_{ip} = 1 | w_p, x_i, \hat \gamma, \hat \zeta_p^\mathrm{EB}, 
%	                                  \hat \beta, \hat \epsilon_i^\mathrm{EB})
%	= \mathrm{logit}^{-1}(
%		w_p' \hat \gamma + \hat \zeta_p^\mathrm{EB} -
%		x_i' \hat \beta - \hat \epsilon_i^\mathrm{EB} )
%.\end{equation}
%For a new response from a new person to a new item, the predicted probably may be
%\begin{equation}
%	\Pr(\tilde y_{i'p'} = 1 | w_{p'}, x_{i'}, \hat \gamma, \hat \beta)
%	= \mathrm{logit}^{-1}(w_{p'}' \hat \gamma -	x_{i'}' \hat \beta)
%,\end{equation}
%where both the residual parts are set to zero.
%The predicted probability for a new person to an existing item is
%\begin{equation}
%	\Pr(\tilde y_{ip'} = 1 | w_p', x_i, \hat \gamma, 
%	                                  \hat \beta, \hat \epsilon_i^\mathrm{EB})
%	= \mathrm{logit}^{-1}(
%		w_p' \hat \gamma -
%		x_i' \hat \beta - \hat \epsilon_i^\mathrm{EB} )
%,\end{equation}
%which is obtained by setting the prediction for $\zeta_{p'}$ to zero. Similarly, the predicted probability for an in-sample person to a new item would be obtained by setting the prediction for $\delta_{i'}$ to zero.


\subsection{Inferences for special cases}

If a special case of the full model is fitted, such as any described in Section~\ref{sec:special-cases}, some predictive inferences may not be available. 
%Specifically, a parameter must be exchangeable (that is, drawn from a distribution) in order to specify a predictive distribution for it. 
For example, with the Rasch model (either the marginal or joint maximum likelihood formulations) the predictive distribution for $\tilde \delta_{i'}$ is unavailable because for the Rasch model $X$ is a series of indicator variables for the existing items and $\epsilon$ and $\tau$ are omitted. By extension, predictive distributions for $\tilde y_{i'p}$ and $\tilde y_{i'p'}$ also cannot be obtained for the Rasch model.


\section{Discussion}

For models that are not hierarchical, frequentist analysis will often be equivalent to a Bayesian analysis using uniform priors. In a more complicated model, like the doubly explanatory model, the results are still expected to be very similar if the priors for Bayesian analysis are uniform or diffuse. Nonetheless, some advantages have been identified in the Bayesian approach.

First, in Bayesian modeling it is natural to make inferences about basic parameters, intermediate parameters, and hyperparameters simultaneously, while frequentist analysis does not directly estimate exchangeable basic parameters or intermediate parameters. Paradoxically, in frequentist item response modeling, the actual measurement of persons, that is obtaining a prediction for $\theta_p$, must occur in a second, post-estimation step when marginal maximum likelihood estimation is used.

Second, Bayesian analysis propagates uncertainty regarding parameters while frequentist analysis does not. For example, frequentist analysis may obtain an empirical Bayes prediction $\hat \zeta_p^\mathrm{EB}$ that will depend on point estimate $\hat \sigma$ and other parameters, and standard errors for $\hat \zeta_p^\mathrm{EB}$ will be unduly small as $\hat \sigma$ is treated as known. In contrast, the Bayesian posterior for $\zeta_p$ will be marginal over the posterior for $\sigma$ (and all other parameters) and so will more accurately represent the uncertainty regarding $\zeta_p$. The difference is more pronounced with an intermediate parameter like $\theta_p$, as the true Bayesian posterior for it will also reflect the uncertainty regarding $\gamma$ in addition to $\zeta_p$.


\printbibliography

\end{document}
